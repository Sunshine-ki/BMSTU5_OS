%\section{ТЗ:}
%Получение с помощью дизассемблера sr.exe кода прераывания INT 8h. 
%Студент составляет отчет, состоящий из полученного кода INT 8h, алгоритма 8-го прерывания и алгоритма подпрограммы, которая вызывается в начале и в конце кода 8-го прерывания, в графическом виде по ГОСТу. Студент защищает работу.

\begin{lstlisting}[label=some-code,caption=Код прерывания INT 8h]
; Вызывает подпрограмму sub_1
020A:0746  E8 0070		;*		call	sub_1			; (07B9)
020A:0746  E8 70 00				db	0E8h, 70h, 00h
; Записывает регистры в стек.
020A:0749  06					push	es
020A:074A  1E					push	ds
020A:074B  50					push	ax
020A:074C  52					push	dx
; Инициализирует регистры.
020A:074D  B8 0040				mov	ax,40h
; В ds помещаем начало области данных BIOS (Зубков).
020A:0750  8E D8				mov	ds,ax
020A:0752  33 C0				xor	ax,ax			; Zero register
; В es помещаем адрес начала таблицы векторов прерывания.
020A:0754  8E C0				mov	es,ax
; 0040:006C = 0046С - адрес 4-байтовой переменной,
; располагающейся в области данных BIOS - это счетчик таймера.
; Увеличивает счетчик таймера.
020A:0756  FF 06 006C			inc	word ptr ds:[6Ch]	; (0040:006C=0A1Dh)
; JNZ - перейти, если не равно (ZF = 0) на loc_1. 
020A:075A  75 04				jnz	loc_1			; Jump if not zero
; Если счетчик равен 0, то увеличиваем часы, т.е. прошел час. ( 0040:006E - это часы)
020A:075C  FF 06 006E			inc	word ptr ds:[6Eh]	; (0040:006E=0Ah)
020A:0760			loc_1:
; Если час не прошел, то сравниваем
; 0040:006E с 24 (это часы 18h == 24)
020A:0760  83 3E 006E 18		cmp	word ptr ds:[6Eh],18h	; (0040:006E=0Ah)
; Если еще не 24, то прыгаем на loc_2 
020A:0765  75 15				jne	loc_2			; Jump if not equal
; Сравниваем 0040:006C (B0h=176) 
020A:0767  81 3E 006C 00B0		cmp	word ptr ds:[6Ch],0B0h	; (0040:006C=0A1Dh)
; Если != 176, то прыгаем на loc_2
020A:076D  75 0D				jne	loc_2			; Jump if not equal
; Обнуляем счетчик (если прошел день)
; В ячейку 0040:0070 записывам единицу
; (Для фиксации о том, что новый день наступил)
020A:076F  A3 006E				mov	word ptr ds:[6Eh],ax	; (0040:006E=0Ah)
020A:0772  A3 006C				mov	word ptr ds:[6Ch],ax	; (0040:006C=0A1Dh)
020A:0775  C6 06 0070 01		mov	byte ptr ds:[70h],1	; (0040:0070=0)
; В младший байт регистра ax заносим 8 
; (т.к. ax до этого был равен 0 => 0 or 8 == 8)
020A:077A  0C 08				or	al,8
020A:077C			loc_2:
; Если новый день не наступил, то
; Записываем регистр ax в стек.
; (Он м.б. равен 0 или 8, в зависимости от того, наступил новый день или нет)
20A:077C  50					push	ax
; Ячейка с адресом 0000:0440h содержит время, оставшееся до выключения двигателя.
; Декрементируем это время. 
020A:077D  FE 0E 0040			dec	byte ptr ds:[40h]	; (0040:0040=2Ch)
; Если еще не равно нулю, то прыгаем на loc_3
020A:0781  75 0B				jnz	loc_3			; Jump if not zero
; Если равено 0, то двигатель НГМД отключается.
; Отправка сигнала отключения моторчика.
; Сброс флага отключания моторчика дисковода
020A:0783  80 26 003F F0		and	byte ptr ds:[3Fh],0F0h	; (0040:003F=0)
020A:0788  B0 0C				mov	al,0Ch
020A:078A  BA 03F2				mov	dx,3F2h
; Порт 3F2 - адрес порта цифрового упарвления (тип вывод).
; НГМД - накопитель на гибких магнитных дисках
; Порт 3F2h работает только на запись, это порт вывода.
; Мы отправляем в этот порт 0C (1100).
; 2 бит поднят - разрешение работы контроллера
; 3 бит поднят - разрешение прерываний и прямого доступа к памяти
; 4-7 биты - значение 1 в каждом разряде вызывает включение соответствующего двигателя НГМД
; (Инф. https://www.frolov-lib.ru/books/bsp/v19/ch1_4.html)
; Инструкция OUT выводит данные из регистра AL или AX (ИСТОЧНИК) в порт ввода-вывода. 
; Номер порта должен быть указан в ПРИЁМНИКЕ.
020A:078D  EE					out	dx,al			; port 3F2h, dsk0 contrl output
020A:078E			loc_3:
; Если счетчик таймера не равен нулю, то
; Возвращаем в ax содержимое, которое раньше положили.
020A:078E  58					pop	ax
; Проверяем флаг PF по адресу 0040:0314. 
; (0100 - поднят 2 бит, он как раз отвечает за флаг PF - Parity Flag - Флаг чётности)
020A:078F  F7 06 0314 0004		test	word ptr ds:[314h],4	; (0040:0314=3200h)
020A:0795  75 0C				jnz	loc_4			; Jump if not zero
; LAHF: Загрузка флагов в регистр АН.  
; Команда LAHF перемещает младший байт регистра флагов EFLAGS в регистр AH.
020A:0797  9F					lahf				; Load ah from flags
; Обмен ah и al.
020A:0798  86 E0				xchg	ah,al
; Записываем ax в стек.
020A:079A  50					push	ax
; Косвенный вызов прерывания 1Ch (1C * 4 = 70h).
020A:079B  26: FF 1E 0070		call	dword ptr es:[70h]	; (0000:0070=6ADh)
020A:07A0  EB 03				jmp	short loc_5		; (07A5)
020A:07A2  90					nop
020A:07A3			loc_4:
; Вызываем прерывание 1C.
; После инициализации системы вектор INT 1Ch указывает на команду IRET, 
; то есть обработчик прерывания INT 1Ch ничего не делает.
020A:07A3  CD 1C				int	1Ch			; Timer break (call each 18.2ms)
020A:07A5			loc_5:
; Вызываем подпрограмму sub_1
020A:07A5  E8 0011				call	sub_1		; (07B9)
; Сброс контроллера прерываний (mov  al, 20h; out  20h, al) - из методички.
; Необходимо отметить, что прерывание int 1Ch вызывается обработчиком прерывания int 8h
; до сброса контроллера прерывания, поэтому во время выполнения
; прерывания int 1Ch все аппаратные прерывания запрещены.
; В частности, запрещены прерывания от клавиатуры.
020A:07A8  B0 20				mov	al,20h			; ' '
; Конец прерывания.
020A:07AA  E6 20				out	20h,al			; port 20h, 8259-1 int command
;  al = 20h, end of interrupt
; Восстанавливаем значение регистров. 
020A:07AC  5A					pop	dx
020A:07AD  58					pop	ax
020A:07AE  1F					pop	ds
020A:07AF  07					pop	es
; Выход.
020A:07B0  E9 FE99				jmp	$-164h
020A:07B3  C4					db	0C4h
;* No entry point to code
; les - загружает первые 16 бит dword по адресу ds:[93E9h] в регистр CX,
; А оставшиеся 16 бит загружает в ES (Т.к. lES (есть еще lDS,...))
020A:07B4  C4 0E 93E9			les	cx,dword ptr ds:[93E9h]	; (0000:93E9=0A1A1h) Load 32 bit ptr
020A:07B8  FE					db	0FEh
\end{lstlisting}

\begin{lstlisting}[label=some-code,caption=Код подпрограммы sub\_1]
				sub_1		proc	near
; Сохраняем флаги.
020A:07B9  1E					push	ds
020A:07BA  50					push	ax
;  Инициализируем регистры.
020A:07BB  B8 0040				mov	ax,40h
020A:07BE  8E D8				mov	ds,ax
; lahf: Загрузка флагов в регистр АН.
; Загружает значение флагового регистра в регистр  АН. 
020A:07C0  9F					lahf				; Load ah from flags
; Команда TEST - логическое и без изменения операда (Меняются только флаги).
; 2400 = 10010000000000. Поднят ли флаг 10ый или 13ый?
; 10 - DF - Direction Flag - Флаг направления. 
; Контролирует поведение команд обработки строк. Если установлен в 1, то строки 
; обрабатываются в сторону уменьшения адресов, если сброшен в 0, то наоборот.
; 12 и 13 - IOPL - I/O Privilege Level - Уровень приоритета ввода/вывода.
020A:07C1  F7 06 0314 2400		test	word ptr ds:[314h],2400h	; (0040:0314=3200h)
; Если не равно 0 переходим на loc_7
020A:07C7  75 0C				jnz	loc_7			; Jump if not zero
; На все время выполнения команды, снабженной таким префиксом, будет заб-
; локирована шина данных, и если в системе присутствует другой процессор, он не
; сможет обращаться к памяти, пока не закончится выполнение команды с префик-
; сом LOCK.
; LOCK - делаеи следующую команду неделимой.
; and 2 раза обращается к памяти. 1 раз он считывает значение по адресу 0040:0314
; Затем он изменяет его и еще раз обращаяется к памяти на запись.
; Мы делаем ее неделимой, чтобы в этот промежуток, когда мы выполняем непосредственно
; Саму логическую операцию, никто не смог влезть в этот участок памяти (мы его как раз блокируем).
020A:07C9  F0> 81 26 0314 FDFF  lock	and	word ptr ds:[314h],0FDFFh	; (0040:0314=3200h)
020A:07D0			loc_6:
; Команда sahf копирует разряды 7, 6, 4, 2 и 0 регистра АН в регистр флагов процессора, 
; устанавливая тем самым значения флагов SF, ZF, AF, PF и CF соответственно. 
; Команда не имеет операндов.
020A:07D0  9E					sahf				; Store ah into flags
; Восстанавливаем флаги.
020A:07D1  58					pop	ax
020A:07D2  1F					pop	ds
020A:07D3  EB 03				jmp	short loc_8		; (07D8)
020A:07D5			loc_7:
; cli - сбрасывает флаг IF
; Флаг IF - Interrupt Enable Flag - Флаг разрешения прерываний.
; Если сбросить этот флаг в 0, то процессор перестанет обрабатывать прерывания от внешних устройств.
; Обычно его сбрасывают на короткое время для выполнения критических участков программы.
; (маскируемые - прерывания, которые можно запрещать установкой соответствующих битов в регистре маскирования прерываний)
020A:07D5  FA					cli				; Disable interrupts
020A:07D6  EB F8				jmp	short loc_6		; (07D0)
020A:07D8			loc_8:
; Конец процесса.
020A:07D8  C3					retn
sub_1		endp
\end{lstlisting}


\begin{figure}[ht!]
	\centering{
		\includegraphics[width=0.6\textwidth]{img/graph1.png}
		\caption{Схема прерывания INT 8h} }
\end{figure}


\begin{figure}[ht!]
	\centering{
		\includegraphics[width=0.4\textwidth]{img/graph2.png}
		\caption{Схема прерывания INT 8h} }
\end{figure}


\begin{figure}[ht!]
	\centering{
		\includegraphics[width=0.6\textwidth]{img/graph3.png}
		\caption{Схема подпрограммы sub\_1} }
\end{figure}